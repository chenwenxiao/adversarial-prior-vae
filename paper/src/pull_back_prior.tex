\section{Pull-back Prior}\label{sec:pull_back_prior}

\subsection{Intuition}
We follow the way of learnable priors and propose Pull-back Prior to improve the representation ability of VAE. It is the basic idea that the optimal solution, aggregated prior, is intractable and non-optimal solution, Pull-back Prior, could lead to better performance than an approximation aggregated prior. The formula of Pull-back Prior is given by:
\begin{equation}\label{eq:pull_back_prior}
	\ln p_\lambda(z) = \ln p_\mathcal{N}(z) - \beta * D(G(z)) - \ln Z \tag{4}
\end{equation}
where $p_\mathcal{N}(z)$ is a simple prior (\EG standard normal) $\beta$ is a scalar called pull-back weight, $D$ is a discriminator defined on $\mathcal{X}$ (data space), $G$ is an generator defined by $G(z) = \E_{p_\theta(x|z)} x$ and $Z$ is the partition function $Z = \int_{\mathcal{Z}} p_\mathcal{N}(z) \exp\{- \beta * D(G(z))\} \dd z$ ($\mathcal{Z}$ denotes latent space).

A simplistic explanation of Pull-back Prior is given following: We would like to get a more powerful prior than simple prior $p_\mathcal{N}$. A simple way is to improve the density of $z$ which generates better data and decrease the density of $z$ which generates worse data. $D$ is a discriminator to assess the quality of $x$. When $D(x)$ is less, $x$ is more similar to real data and of higher quality. We could pull-back the discriminator from data space to latent space, and function $D(G(z))$ represents the quality of the data generated by $z$. To improve and decrease the density at better $z$ and worse $z$, we modify $p_\mathcal{N}(z)$ by $\beta * D(G(z))$ and then normalize it by $Z$, and finally we obtain the Pull-back Prior. 

\subsection{Inference}~\label{subsec:inference}

Before the starting of inference of Pull-back Prior, we need to review the inference of aggregated posterior. We divide the optimization of $\min_{\theta, \phi, \lambda} \mathcal{L}(\theta, \phi, \lambda)$ into 2 part $\min_{\theta, \phi} \min_{\lambda} \mathcal{L}(\theta, \phi, \lambda)$. Considering the 2nd optimization $\min_\lambda \mathcal{L}(\theta, \phi, \lambda)$, the optimal solution is $p_\lambda(z) = q_\phi(z)$. The key idea of the inference of Pull-back Prior is to set another objective function $\hat{\mathcal{L}}$ for 2nd optimization. Noticing that the ELBO is derived by KL-divergence between $p^*$ and $p_\theta$, a candidate $\hat{\mathcal{L}}$ could be another divergence. This operation is called Double Metrics Analysis (DMA). 

Choosing another divergence will lead to a new learnable prior rather than $q_\phi$. We could make this new learnable prior feasible and efficient, but however, it will never be the theoretical optimal prior, \IE the essence of DMA is to get an acceptable trade-off between theory and practice. 

We choose Wasserstein distance for 2nd optimization, because it shows wonderful performance in WGAN and has stable theoretical basis in transition theory. Considering following optimization:
\begin{align*}\label{eq:direct_2nd_optimization}
	& \min_{\lambda} \hat{\mathcal{L}}(\theta, \phi, \lambda) = \min_{\lambda} W^1(p_\theta, p^*) = \\
	& \min_{\lambda} \sup_{Lip(D) \leq 1} \{\E_{p_\lambda(z)} \E_{p_\theta(x|z)} D(x)  - \E_{p^*(x)} D(x)\} 
	\tag{5}
\end{align*}

It is hard to get an analytical solution of $\lambda$ directly from \cref{eq:direct_2nd_optimization}, therefore we add an assumption to simplify it. Since $p_\theta(x|z)$ is usually a distribution with small variance, it is rational to assume $\E_{p_\theta(x|z)} D(x) = D(\E_{p_\theta(x|z)} x) = D(G(z))$. Even though, the optimization $\sup_{Lip(D) \leq 1}$ is still tough because we need find optimal $D$ for each $\lambda$. If we restrict $p_\lambda$ near the $p_\mathcal{N}$, this optimization may be approximated by a fixed $D$ obtained in $W^1(p^\dag, p^*)$, where $p^\dag(x) = \E_{p_\mathcal{N}(z)}p_\theta(x|z)$ (distribution generated by $p_\mathcal{N}$). Consequently, the simplified optimization is following:
\begin{align*}\label{eq:final_optimization}
	& \min_{\lambda} \{\E_{p_\lambda(z)} D(G(z))  - \E_{p^*(x)} D(x)\} \tag{6} \\
	{\textbf{s.t. }} & KL(p_\lambda, p_\mathcal{N}) \leq \alpha, \qquad \int_{\mathcal{Z}} p_\lambda(z) \dd z = 1, \\
	& D = \arg \sup_{Lip(D) \leq 1} \{\E_{p_\mathcal{N}(z)} D(G(z))  - \E_{p^*(x)} D(x)\}
\end{align*}

We could solve the simplified optimization \cref{eq:final_optimization} by Lagrange multiplier method introduced by calculus of variation~\cite{gelfand2000calculus}. The Lagrange function with Lagrange multiplier $\eta, \gamma$ is following:
\begin{align*}\label{eq:lagrange_function}
& F(p_\lambda, \eta, \gamma) = \E_{p_\lambda(z)} D(G(z))  - \E_{p^*(x)} D(x) + \\
& \eta (\int_{\mathcal{Z}} p_\lambda(z) \dd z - 1) + \gamma(KL(p_\lambda, p_\mathcal{N}) - \alpha) \tag{7}
\end{align*}

We solve \cref{eq:lagrange_function} by Euler-Lagrange equation:
\begin{equation*}\label{eq:euler_lagrange_eqaution}
	D(G(z)) + \eta - \gamma (\ln p_\lambda(z) + 1 - \ln p_\mathcal{N}(z)) = 0 \tag{8}
\end{equation*}
Therefore, $\ln p_\lambda(z) = \frac{1}{\gamma} D(G(z)) + \ln p_\mathcal{N}(z) + (\frac{\eta}{\gamma} - 1)$ is the optimal solution, which could be organized into \cref{eq:pull_back_prior}. From this inference, we could explain the meaning of $\beta = \frac{1}{\gamma}$ and $Z = \frac{\eta}{\gamma} - 1$. $\beta$ represents how far $p_\lambda$ is from $p_\mathcal{N}$, since $\gamma$ is the Lagrange multiplier of constraint $KL(p_\lambda, p_\mathcal{N}) \leq \alpha$. $Z$ is the partition function since $\eta$ is the Lagrange multiplier of constraint $\int_{\mathcal{Z}} p_\lambda(z) \dd z = 1$.

We obtain the basic formula of Pull-back Prior, which is an extension of $p_\mathcal{N}$, who is the special case that $\beta = 0$. However, it remains some troubles about how to optimize it and calculate partition function $Z$ in VAE architecture. 



