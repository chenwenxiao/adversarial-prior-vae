\section{Background}

\subsection{VAE and learnable prior}

Many generative models aim to minimize the KL-divergence between empirical distribution $p^*(x)$ and model distribution $p_\theta(x)$, which lead to the maximization of log-likelihood. VAE models the joint distribution $p_\theta(x, z)$ where $Z$ is a latent variable and the marginal distribution is $p_\theta(x) = \int p_\theta(x, z) \dd z$. VAE apply variational inference to obtain the evidence lower bound objective (ELBO): 
\begin{align*} \label{eq:ELBO}
\ln p_\theta(x) \geq \E_{q_\phi(z|x)} [  &  \ln p_\theta(x|z) + \ln p_\theta(z) - \\ & \ln q_\phi(z|x) ] 
\triangleq \mathcal{L}(x; \theta, \phi) \tag{1}
\end{align*}
where $q_\phi(z|x)$ is the variational posterior (encoder) and $p_\theta(x|z)$ is the true posterior (decoder). The training objective of VAE is $\EEE{p^*(x)}{\mathcal{L}(x; \theta, \phi)}$ and it is optimized by SGVB with re-parameterization trick. In vanilla VAE, prior $p_\theta(z)$ is chosen as the standard normal distribution. 

Recently, some researchers realized that the simplistic prior could lead to poor hidden representation and many learnable priors are proposed subsequently to improve the representation ability of prior. The learnable prior is denoted by $p_\lambda(z)$. Most of them focus on the aggregated posterior $q_\phi(z)$, which is shown as the optimal prior for ELBO by following decomposition:
\begin{align*} \label{eq:second-decomposition}
\mathcal{L}(\theta, \phi, \lambda) = \E_{p^*(x)}\E_{q_\phi(z|x)} [ \ln p_\theta(x|z)] + \\ 
\E_{p^*(x)}[\mathbb{H}[q_\phi(z|x)]] + \E_{q_\phi(z)} \ln p_\lambda(z) \tag{2}
\end{align*}
where $p_\lambda(z)$ only appears in the last term and the optimal solution of $p_\lambda(z)$ is $q_\phi(z)$. 

\subsection{GAN and Wasserstein distance}

A series of GANs are proposed in few years, such as (). The key idea of vanilla GAN is that train a generator to generate samples to deceive discriminator, and a discriminator to distinguish the generated samples and real samples. However, vanilla is unstable in training process and WGAN and Wasserstein distance is introduced for tackling this problem. Wasserstein distance is based on transition theory and it vastly extent the theory of GAN. 1-st Wasserstein distance $W^1(\mu, \nu)$ is used for calculating the distance between two measures $\mu, \nu$. The dual form of Wasserstein distance is following:
\begin{align*} \label{eq:Wasserstein-distance}
W^1(\mu, \nu) = \sup_{Lip(D) \leq 1} \{\E_{\mu(x)} D(x)  - \E_{\nu(x)} D(x)\} \tag{3}
\end{align*}
where $Lip(D) \leq 1$ means $D$ is 1-Lipschitz. WGAN is optimized by minimizing $W^1(p^*, p_\theta)$ which can be expanding to a min-max optimization, whose parameters are $D$ and $\theta$. 

WGAN makes progress toward stable training but sometimes fails to converge due to the use of weight clipping to enforce the Lipschitz constrain. WGAN-GP points out this issue and improves WGAN by gradient penalty technique to implement more stable training and WGAN-div proposes alternative method of gradient penalty. These methods make WGAN framework become more robust and stable. 
